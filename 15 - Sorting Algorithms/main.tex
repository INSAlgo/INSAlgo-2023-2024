\documentclass[compress,12pt,bookmark]{beamer}

\usepackage{booktabs}
%\usepackage{verbatim}

% packages for code
\usepackage{minted}

\usepackage{pgfplots}
\pgfplotsset{compat=1.18}
\usetikzlibrary{arrows.meta}
\usepackage{xcolor}

% Define custom colors
\definecolor{CustomCyan}{RGB}{0, 183, 235}
\definecolor{CustomGreen}{RGB}{0, 178, 89}
\definecolor{CustomYellow}{RGB}{220, 190, 0}
\definecolor{CustomOrange}{RGB}{255, 135, 0}
\definecolor{CustomRed}{RGB}{255, 30, 0}
\definecolor{CustomPurple}{RGB}{128, 0, 128}
\definecolor{CustomBlack}{RGB}{20, 20, 20}

\usetheme{Arguelles}

\title{Sorts}
\subtitle{Understanding Array Sorting Algorithms, Efficiency, Comparison and Implementation}
\event{INSAlgo}
\date{\today}
\author{Onyr (Florian RASCOUSSIER)}
\institute{INSA Lyon \& IMT Atlantique}
\email{florian.rascoussier@insa-lyon.fr}
%\homepage{}
\github{0nyr}

\begin{document}

\frame[plain]{\titlepage}

\section{Introduction}

\begin{frame}[plain,standout]
    \centering
    \Huge
    Introduction

    \normalsize
    Discussing complexity, and efficiency of algorithms.
    \vfill
\end{frame}

\begin{frame}
    \frametitle{Runtime and Memory Complexity: Basics}

    \begin{itemize}
          \item \textbf{Runtime Complexity}
                \begin{itemize}
                      \item Time an algorithm takes relative to input length.
                \end{itemize}
          \item \textbf{Memory Complexity}
                \begin{itemize}
                      \item Memory needed by an algorithm relative to input size.
                \end{itemize}
          \item Both are crucial for:
                \begin{itemize}
                      \item Comparing algorithm efficiency.
                      \item Choosing the right algorithm for the job.
                \end{itemize}
    \end{itemize}
\end{frame}

\begin{frame}
    \frametitle{Measuring Program Execution Time: Challenges}
    \framesubtitle{Types of Time Measurements}

    \begin{itemize}
        \item \textbf{Time Measurement Issues}
              \begin{itemize}
                  \item Real Time: Wall-clock time for program execution.
                  \item User Time: CPU time for executing user program code.
                        \begin{itemize}
                            \item Excludes system operations.
                            \item Reflects direct program execution time.
                        \end{itemize}
                  \item System Time\footnote{See \url{https://stackoverflow.com/questions/556405/what-do-real-user-and-sys-mean-in-the-output-of-time1}}: CPU time for system operations for the program.
                        \begin{itemize}
                            \item File operations, I/O tasks.
                            \item Essential for resource-intensive operations.
                        \end{itemize}
              \end{itemize}
        \item Variability in measurements due to:
              \begin{itemize}
                  \item System load, resources, hardware.
                  \item Inconsistencies across environments.
              \end{itemize}
    \end{itemize}
\end{frame}

\begin{frame}
    \frametitle{Measuring Program Execution Time: Theoretical Approach}
    \framesubtitle{Abstracting Time Measurements}

    \begin{itemize}
        \item \textbf{Theoretical Approach}
              \begin{itemize}
                  \item Approximate with \textbf{input size} (n) and \textbf{operation count}.
                  \item $n \in \mathbb{N}^*$: Number of loops or iterations -- main driver of complexity.
                  \item $k \in \mathbb{N}^*$: Parameters affecting complexity, aside from input size.
                        \begin{itemize}
                            \item Here intended as range of the non-negative key values
                        \end{itemize}
                  \item Focus on growth trends rather than exact times.
              \end{itemize}
        \item \textbf{Conclusion:}
              \begin{itemize}
                  \item Theoretical focus helps identify scalability issues.
                  \item Prioritizes relative efficiency over absolute timing.
              \end{itemize}
    \end{itemize}
\end{frame}

\section{Comparing Algorithms}

\begin{frame}[plain,standout]
    \centering
    \Huge
    Comparing Algorithms

    \normalsize
    Big O notation, visualization, and classic sorting algorithms.
    \vfill
\end{frame}

\begin{frame}
    \frametitle{Understanding Big O Notation}
    \framesubtitle{A teaspoon of math}

    \begin{itemize}
        \item \textbf{Big O Notation}: Describes the upper bound of complexity.
              \begin{itemize}
                  \item Focuses on worst-case scenario.
                  \item Ignores constant factors and lower order terms.
              \end{itemize}
        \item \textbf{Basic Rules}
              \begin{itemize}
                  \item \textit{Linear Terms}: $\mathcal{O}(\alpha n + \beta) = \mathcal{O}(n)$.
                        \begin{itemize}
                            \item Constants $\alpha$, $\beta$ don't affect growth rate.
                        \end{itemize}
                  \item \textit{Sum Rule}: $\mathcal{O}(f(n)) + \mathcal{O}(g(n)) = \mathcal{O}(\max(f(n), g(n)))$.
                  \item \textit{Product Rule}: $\mathcal{O}(f(n)) \cdot \mathcal{O}(g(n)) = \mathcal{O}(f(n) \cdot g(n))$.
              \end{itemize}
        \item \textbf{Implications}
              \begin{itemize}
                  \item Simplifies comparing algorithms.
                  \item Emphasizes dominant factors affecting growth.
              \end{itemize}
        \item \textbf{Example}
              \begin{itemize}
                  \item $\mathcal{O}(3n^2 + 10n + 100) = \mathcal{O}(n^2)$.
                        \begin{itemize}
                            \item $n^2$ term dominates as $n$ grows.
                        \end{itemize}
              \end{itemize}
    \end{itemize}
\end{frame}

\begin{frame}[fragile]
    \frametitle{Visualization}
    \framesubtitle{A picture is worth a \texttt{0x3E8} words}

    \def\stretchfactor{1.5}
    \begin{tikzpicture}[xscale=\stretchfactor, every node/.style={font=\small}]

        % Grid and Axes
        \draw[very thin,color=gray] (-0.1,0.1) grid (5.9,5.9);
        \draw[->] (-0.2,0) -- (6.2,0) node[right] {$n$};
        \draw[->] (0,-0.2) -- (0,6.2) node[above] {$f(n)$};

        % axis label
        \draw (3,0) node[below] {input size};
        \draw (0,3) node[rotate=90, above] {impact on computation time};

        % display a dot at (0, 0)
        \fill[black] (0,0) circle (0.05);
      
        % Constant Time (O(1))
        \draw[color=CustomCyan, domain=0.0:6.0, thick] plot(\x, 0) node[right, above] {$O(1)$};
      
        % Logarithmic Time (O(log n))
        \draw[color=CustomGreen, domain=0:6.0, thick] plot(\x, {ln(\x + 1)}) node[right, right] {$O(\log n)$};

        % Linear Time (O(n))
        \draw[color=CustomYellow, domain=0.0:6.0, thick] plot (\x, \x) node[right, right] {$O(n)$};

        % Linearithmic Time (O(n log n))
        \draw[color=CustomOrange, domain=0:2.8, thick] plot(\x, {\x * ln(\x + 2)/ln(2)}) node[right, right] {$O(n \log n)$};

        % Quadratic Time (O(n^2))
        \draw[color=CustomRed, domain=0.0:2.07, thick] plot (\x, {(\x + 1) * \x}) node[right, right] {$O(n^2)$};

        % Cubic Time (O(n^3))
        \draw[color=CustomPurple, domain=0.0:1.25, thick] plot (\x, {(\x + 1) * (\x + 1) * \x}) node[right, right] {$O(n^3)$};

        % Exponential Time (O(2^n))
        \draw[color=CustomBlack, domain=0:0.636, thick] plot (\x, {pow(2, \x + 1) * (\x + 1) * (\x + 1) - 2}) node[right, right] {$O(2^n)$};
        
      \end{tikzpicture}
\end{frame}


\begin{frame}
    \frametitle{Classic Sorting Algorithms and Their Complexities}
    \framesubtitle{Know your classics!}
    \begin{table}
        \begin{tabular}{lcccc}
            \toprule
            Algo. & Best & Average & Worst & Mem. \\
            \midrule
            Selection Sort & $\mathcal{O}(n^2)$ & $\mathcal{O}(n^2)$ & $\mathcal{O}(n^2)$ & $\mathcal{O}(1)$ \\
            Insertion Sort & $\mathcal{O}(n)$ & $\mathcal{O}(n^2)$ & $\mathcal{O}(n^2)$ & $\mathcal{O}(1)$ \\
            Bubble Sort & $\mathcal{O}(n)$ & $\mathcal{O}(n^2)$ & $\mathcal{O}(n^2)$ & $\mathcal{O}(1)$ \\
            Merge Sort & $\mathcal{O}(n \log n)$ & $\mathcal{O}(n \log n)$ & $\mathcal{O}(n \log n)$ & $\mathcal{O}(n)$ \\
            Quick Sort & $\mathcal{O}(n \log n)$ & $\mathcal{O}(n \log n)$ & $\mathcal{O}(n^2)$ & $\mathcal{O}(\log n)$ \\
            \bottomrule
        \end{tabular}
    \end{table}
\end{frame}

\begin{frame}
    \frametitle{More Algorithms and Their Complexities\footnote{See \url{https://www.bigocheatsheet.com/}}}
    \begin{table}
        \scriptsize % Adjust font size to fit the table in the frame
        \begin{tabular}{lcccc}
            \toprule
            Algorithm & Best & Average & Worst & Memory \\
            \midrule
            Selection Sort & $\Omega(n^2)$ & $\Theta(n^2)$ & $O(n^2)$ & $O(1)$ \\
            Insertion Sort & $\Omega(n)$ & $\Theta(n^2)$ & $O(n^2)$ & $O(1)$ \\
            Bubble Sort & $\Omega(n)$ & $\Theta(n^2)$ & $O(n^2)$ & $O(1)$ \\
            Merge Sort & $\Omega(n \log n)$ & $\Theta(n \log n)$ & $O(n \log n)$ & $O(n)$ \\
            Quick Sort & $\Omega(n \log n)$ & $\Theta(n \log n)$ & $O(n^2)$ & $O(\log n)$ \\
            Timsort & $\Omega(n)$ & $\Theta(n \log n)$ & $O(n \log n)$ & $O(n)$ \\
            Heapsort & $\Omega(n \log n)$ & $\Theta(n \log n)$ & $O(n \log n)$ & $O(1)$ \\
            Tree Sort & $\Omega(n \log n)$ & $\Theta(n \log n)$ & $O(n^2)$ & $O(n)$ \\
            Shell Sort & $\Omega(n \log n)$ & $\Theta(n(\log n)^2)$ & $O(n(\log n)^2)$ & $O(1)$ \\
            Bucket Sort & $\Omega(n+k)$ & $\Theta(n+k)$ & $O(n^2)$ & $O(n)$ \\
            Radix Sort & $\Omega(nk)$ & $\Theta(nk)$ & $O(nk)$ & $O(n+k)$ \\
            Counting Sort & $\Omega(n+k)$ & $\Theta(n+k)$ & $O(n+k)$ & $O(k)$ \\
            Cubesort & $\Omega(n)$ & $\Theta(n \log n)$ & $O(n \log n)$ & $O(n)$ \\
            \bottomrule
        \end{tabular}
    \end{table}
\end{frame}

\section{Tips and tricks}

\begin{frame}[plain,standout]
    \centering
    \Huge
    Tips and Tricks

    \normalsize
    Understanding the basics of sorting algorithms, and how to improve them.
    \vfill
\end{frame}

\begin{frame}[fragile]
    \frametitle{Introducing Bubble Sort}
    \framesubtitle{Humble start}

    \begin{itemize}
        \item \textbf{How It Works}
        \begin{itemize}
            \item Repeatedly steps through the list, compares adjacent elements, and swaps them if they are in the wrong order.
            \item Continues until no more swaps are needed.
            \item It works? Yes. Each inner loop iteration moves elements by one position. In the worst case, it needs \(n\) iterations to move the leftmost element to the rightmost position.
        \end{itemize}
        \item \textbf{Complexity: \(\mathcal{O}(n^2)\)}
    \end{itemize}

    \rule{\textwidth}{1pt}
    \scriptsize
    \begin{minted}{python}
        def bubble_sort(arr):
            n = len(arr)
            for _ in range(n):
                for j in range(n - 1):
                    if arr[j] > arr[j + 1]:
                        arr[j], arr[j + 1] = arr[j + 1], arr[j] # Swap
            return arr
    \end{minted}
    \rule{\textwidth}{1pt}

\end{frame}

\begin{frame}[fragile]
    \frametitle{Improving Bubble Sort}
    \framesubtitle{Stay DRY!}

    \begin{itemize}
        \item \textbf{Optimizing Bubble Sort}
            \begin{itemize}
            \item \textit{Early Termination}: Stop if no swaps are made in an iteration.
            \item \textit{Reduced Iterations}: Reduce the number of iterations as the array becomes sorted.
            \end{itemize}
        \item \textbf{Complexity: \(\mathcal{O}(n^2)\)}
            \begin{itemize}
            \item Best case: \(\mathcal{O}(n)\) for already sorted arrays.
            \end{itemize}
    \end{itemize}

    \rule{\textwidth}{1pt}
    \scriptsize
    \begin{minted}{python}
        def bubble_sort(arr):
            n = len(arr)
            for i in range(n):
                swapped = False
                for j in range(i, n - 1):
                    if arr[j] > arr[j + 1]:
                        arr[j], arr[j + 1] = arr[j + 1], arr[j] # Swap
                        swapped = True
                if not swapped:
                    break
            return arr
    \end{minted}
    \rule{\textwidth}{1pt}

\end{frame}


\section{Reviewing Classics}

\begin{frame}[plain,standout]
    \centering
    \Huge
    Reviewing Classics

    \normalsize
    Selection, Insertion, Bubble, Merge, and Quick Sort.
    \vfill
\end{frame}

\begin{frame}[fragile]
    \frametitle{Selection Sort Algorithm}
    \framesubtitle{Simplicity in Action}

    \begin{itemize}
        \item \textbf{How It Works}
              \begin{itemize}
                  \item Iteratively selects the smallest element from the unsorted portion and swaps it with the element at the current position.
                  \item Continues until the entire array is sorted.
              \end{itemize}
    \end{itemize}

    \rule{\textwidth}{1pt}
    \scriptsize
    \begin{minted}{python}
        def selection_sort(arr):
            for i in range(len(arr)):
                min_idx = i
                for j in range(i+1, len(arr)):
                    if arr[j] < arr[min_idx]:
                        min_idx = j
                arr[i], arr[min_idx] = arr[min_idx], arr[i]
            return arr
    \end{minted}
    \rule{\textwidth}{1pt}
\end{frame}

\begin{frame}[fragile]
    \frametitle{Insertion Sort Algorithm}
    \framesubtitle{Building the Sorted Array}

    \begin{itemize}
        \item \textbf{How It Works}
              \begin{itemize}
                  \item Builds the sorted array one element at a time.
                  \item Iterates through the array, shifting elements to the right to make space for the current element.
              \end{itemize}
    \end{itemize}

    \rule{\textwidth}{1pt}
    \scriptsize
    \begin{minted}{python}
        def insertion_sort(arr):
            for i in range(1, len(arr)):
                key = arr[i]
                j = i-1
                while j >= 0 and key < arr[j]:
                    arr[j+1] = arr[j]
                    j -= 1
                arr[j+1] = key
            return arr
    \end{minted}
    \rule{\textwidth}{1pt}
\end{frame}

\begin{frame}[fragile]
    \frametitle{Bubble Sort Algorithm}
    \framesubtitle{Bubbling Up the Largest Element}

    \begin{itemize}
        \item \textbf{How It Works}
              \begin{itemize}
                  \item Repeatedly steps through the list, compares adjacent elements and swaps them if they are in the wrong order.
                  \item Continues until no more swaps are needed.
              \end{itemize}
    \end{itemize}

    \rule{\textwidth}{1pt}
    \scriptsize
    \begin{minted}{python}
        def bubble_sort(arr):
            n = len(arr)
            for i in range(n):
                for j in range(0, n-i-1):
                    if arr[j] > arr[j+1]:
                        arr[j], arr[j+1] = arr[j+1], arr[j]
            return arr
    \end{minted}
    \rule{\textwidth}{1pt}
\end{frame}

\begin{frame}[fragile]
    \frametitle{Quicksort Algorithm}
    \framesubtitle{Divide and Conquer}

    \begin{itemize}
        \item \textbf{How It Works}
              \begin{itemize}
                  \item Employs a divide-and-conquer strategy to sort the array.
                  \item Selects a 'pivot' element and partitions the array into sub-arrays of elements less than and greater than the pivot.
                  \item Recursively applies the same strategy to the sub-arrays.
                  \item Highly efficient for large datasets, with average time complexity of O(n log n).
              \end{itemize}
    \end{itemize}

    \rule{\textwidth}{1pt}
    \scriptsize
    \begin{minted}{python}
        def quicksort(arr):
            if len(arr) <= 1:
                return arr
            pivot = arr[len(arr) // 2]
            left = [x for x in arr if x < pivot]
            middle = [x for x in arr if x == pivot]
            right = [x for x in arr if x > pivot]
            return quicksort(left) + middle + quicksort(right)
    \end{minted}
    \rule{\textwidth}{1pt}
\end{frame}


\begin{frame}[fragile]
    \frametitle{Merge Sort Algorithm}

    \begin{itemize}
        \item \textbf{How It Works}
              \begin{itemize}
                  \item Another divide-and-conquer algorithm that divides the array into halves, sorts each half, and then merges the sorted halves together.
                  \item Begins with the division of the array into smallest manageable units, then merges units in a sorted manner to form larger sorted sections until the whole array is merged back together.
                  \item Consistently performs with a time complexity of O(n log n), making it highly efficient for large data sets.
              \end{itemize}
        \item \textbf{Key Features}
              \begin{itemize}
                  \item Predictable performance.
                  \item Stable sorting algorithm.
                  \item Popular. Default sorting algorithm for many programming languages, but tends to be replaced by Timsort in practice.
              \end{itemize}
    \end{itemize}
\end{frame}

\begin{frame}[fragile]
    \frametitle{Merge Sort Algorithm}

    \scriptsize
    \begin{minted}{python}
        def merge_sort(arr):
            if len(arr) > 1:
                mid = len(arr) // 2
                L = arr[:mid]
                R = arr[mid:]
                merge_sort(L)
                merge_sort(R)
                i = j = k = 0
                while i < len(L) and j < len(R):
                    if L[i] < R[j]:
                        arr[k] = L[i]
                        i += 1
                    else:
                        arr[k] = R[j]
                        j += 1
                    k += 1
                while i < len(L):
                    arr[k] = L[i]
                    i, k = i + 1, k + 1
                while j < len(R):
                    arr[k] = R[j]
                    j, k = j + 1, k + 1
            return arr
    \end{minted}
\end{frame}




\begin{frame}
    \frametitle{Thank You for Your Attention!}
    \framesubtitle{Further Resources}

    \begin{itemize}
        \item \textbf{Useful Links}
              \begin{itemize}
                  \item Big O Cheat Sheet: \url{https://www.bigocheatsheet.com/}
                        \begin{itemize}
                            \item A handy reference for complexity of common data structures and algorithms.
                        \end{itemize}
                  \item Sorting Algorithms with Animations: \url{https://www.toptal.com/developers/sorting-algorithms/bubble-sort}
                        \begin{itemize}
                            \item Explore how different sorting algorithms work with interactive animations.
                        \end{itemize}
              \end{itemize}
        \item \textbf{Contact \& Feedback}
              \begin{itemize}
                  \item My GitHub: \url{https://github.com/0nyr}
                  \item This presentation: \url{https://github.com/0nyr/sorting_algorithms}
              \end{itemize}
    \end{itemize}

    \centering{Once again, thank you and have a great day!}
\end{frame}


\end{document}
